\section{(11.11, 11:20)}
\subsection{Столыпинская аграрная реформа}
1861 - отмена крепостного права - главная незавершенность: земля, выкупленная крестьянами у помещиков, переходит не в частную собственность крестьян, а в собственность крестьянской общины, чтобы ввести в ней круговую поруку в области уплаты налога.

Незапланированным результатом этого стал демографический взрыв, поскольку общинники наделялись землей по уравненному принципу - аграрное перенаселение и крестьянское малоземелье.

По итогам первой всеобщей переписи 1897 года, в стране насчитывалось около 20 миллионов лишних (малоземельных) крестьян - социальная напряженность деревни резко возрастает, что приводит к Первой русской революции 1905-1907 годов.

\subsection{Механизм превращения страны из аграрной в индустриальную}
\begin{enumerate}
	\item Необходимы два условия: личная свобода крестьян и право их частной собственности на землю.
	\item Запускается механизм экономической дифференциации крестьян: появляется спрос на продукцию легкой промышленности, так как богатые могут себе это позволить, а бедным деваться некуда.
	\item Легкая промышленность формирует спрос на машиностроение, а оно вытягивает за собой все остальные отрасли тяжелой промышленности.
	\item Внутренний рынок расширяется, грузопотоки растут - развивается система коммуникации и новые виды транспорта, что обеспечивает мощный толчок к развитию тяжелой промышленности.
\end{enumerate}

\paragraph{Вывод} Один из тормозов остается - общинное землевладение тормозит развитие, и механизм не запускается.

1906 год - предсовмин России Петр Аркдьевич Столыпин - впревые в российской истории пытается провести реформу, нацеленную на повышение покупательной способности населения - аграрная реформа.

\textbf{Цели и задачи реформы:} 
\begin{enumerate}
	\item Ликвидировать последний тормоз на пути к индустриализации;
	\item Создать в России мощный средний класс как опору государства и препятствие революции.
\end{enumerate}

\paragraph{Механизм реформы}
\begin{enumerate}
	\item Разваливая кретьянскую общину, осенью 1906 года с бесплатной передачей общинных зеель в частную собственность. Тем самым запуская экономическую дифференциацию.
	\item Богатым крестьянам даем льготные кредиты через специальный Крестьянский поземельный банк (чтобы скупали земли своих верных односельчан).
	\item Бедных переселяем в Сибирь и на Дальний Восток, создавая все необходимые условия для развития переселенцев: переезд бесплатный, выдача безвозвратного единовременного пособия, на новом месте на семью выделяется в частную собственность бесплатно 60 га земли, организуются льготные ссуды + освобождение от налогов и воинской повинности.
\end{enumerate}

Реформа проваливается из-за элементарной нехватки времени - Столыпин как минимум отводил на реформу 20 лет, но не прошло и восьми лет, как начинается Первая мировая война.

\subsection{Причины Первой мировой войны}
1904-1905 года - русско-японская война: мы проигрываем и оказыаемся в изоляции.

Англия - наш пока потенциальный соперник - не простила нам Персии.

Франция - наш "холодный" \ союзник - образовали с ней союз с в 1893 году, но в 1904 году Франция помирилась с Англией и стала ориентироваться на неё.

Япония - желает продолжения.

Австро-Венгрия - как и мы, претендует на Балканы - Балканский вопрос станет одной из причин Первой мировой.

Германия - желает колонии, в том числе и английскую Индию, к которой собирается добираться по суше - железная дорога Драйбе (Берлин - Белград - Багдад). Важнейшая часть магистрали должна пройти через Турцию - Германия усиливает там свои позиции, что угрожает нашим интересам по Черноморским проливам. Кроме того, выходя на Багдад (Ирак), немцы угрожают нашим интересам в нашей Персии (Иране (с 1926 года)).

Придется идти на уступки - проще всего Англии: в 1907 году сдаем ей Южный и Центральный Иран - Англия нас любит, Франция нас тоже любит, Япония под давлением Англии и Франции (кредиторы Японии) тоже нас любит. В том же 1907 году создается Антанта в составе России, Англии и Франции.

Главная проблема с Германией - как выставить её из Турции, а саму Турцию превратить в нашего союзника?

Помогают итальянцы (союзники Германии), которые в 1911 году нападают на Турцию (союзник Германии).

Итог: Германия позволила Италии покусать Турцию, у Турции сомнения в надежности Германии. Мы этим пользуемся и в 1912 году создаем Балканский союз в составе Болгарии, Сербии, Греции. 

\paragraph{Цели союза:}
\begin{enumerate}
	\item оборонительная: противодействовать Австрии, если та захочет напасть на Сербию;
	\item пугательная: напугать Турцию, чтобы та отвернулась от Германии и перешла к нам.
\end{enumerate}