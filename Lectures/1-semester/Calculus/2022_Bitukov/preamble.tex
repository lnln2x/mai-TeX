\documentclass[a4paper, 12pt]{article}

% Поддержка русского языка
\usepackage[english, russian]{babel}
\usepackage[T2A]{fontenc}
\usepackage[utf8]{inputenc}
\usepackage{indentfirst}

\usepackage{amsmath, amsfonts, amssymb, amsthm, mathtools}

%% Оформление страницы
\usepackage{extsizes}     % Возможность сделать 14-й шрифт
\usepackage{geometry}     % Простой способ задавать поля
\usepackage{setspace}     % Интерлиньяж
\usepackage{enumitem}     % Настройка окружений itemize и enumerate
\setlist{leftmargin=25pt} % Отступы в itemize и enumerate

\geometry{top=25mm}    % Поля сверху страницы
\geometry{bottom=30mm} % Поля снизу страницы
\geometry{left=20mm}   % Поля слева страницы
\geometry{right=20mm}  % Поля справа страницы

\setlength\parindent{15pt}        % Устанавливает длину красной строки 15pt
\linespread{1.3}                  % Коэффициент межстрочного интервала
%\setlength{\parskip}{0.5em}      % Вертикальный интервал между абзацами
%\setcounter{secnumdepth}{0}      % Отключение нумерации разделов
%\setcounter{section}{-1}         % Нумерация секций с нуля
\usepackage{multicol}			  % Для текста в нескольких колонках
\usepackage{soulutf8}             % Модификаторы начертания

% Титульный лист
\newcommand{\CourseName}{Математический анализ}
\newcommand{\FullCourseName}{МАТЕМАТИЧЕСКИЙ АНАЛИЗ}
\newcommand{\SemesterNumber}{I}
\newcommand{\Lecturer}{Юрий Иванович Битюков}
\newcommand{\CourseDate}{осень 2022}

\usepackage{titleps}
\newpagestyle{main}{
	\setheadrule{0.1pt}
	\sethead{\CourseName}{}{}
	\setfootrule{0.1pt}
	\setfoot{ПМИ МАИ, \CourseDate}{}{\thepage}
}
\pagestyle{main}

% Для теорем (разметки текстового материала)
\theoremstyle{plain}
\newtheorem{theorem}{Теорема}[section]
\newtheorem{lemma}{Лемма}[section]
\newtheorem{proposition}{Утверждение}[section]

\theoremstyle{definition}
\newtheorem*{definition}{Определение}
\newtheorem*{myax}{Аксиома}
\newtheorem*{corollary}{Следствие}
\newtheorem*{note}{Замечание}
\newtheorem*{example}{Пример}

\theoremstyle{remark}
\newtheorem*{solution}{Решение}