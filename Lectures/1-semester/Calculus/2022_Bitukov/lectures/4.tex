\section{Предел последовательности. Общие свойства предела. Арифметические свойства сходящихся последовательностей. Предельный переход в неравенствах. Бесконечно малые и бесконечно большие последовательности, их свойства}

\begin{definition}
	Число $A$ называется пределом числовой последовательности $\{x_n\}_{n \in \mathbb{N}}$, если для $\forall \epsilon > 0 \exists n_\epsilon \in \mathbb{N}: \forall n > n_{\epsilon} \Rightarrow |x_n - A| < \epsilon$.
\end{definition}