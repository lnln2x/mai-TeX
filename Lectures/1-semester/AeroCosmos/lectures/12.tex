\section{}

\subsection{}

$p(y) = p_o e^{-\frac{m_o g}{ki}}$ - параметрическая формула, как изменяется атмосфреное давление с высотой.

Можно сказать, что каждые пять километров в высоту давление уменьшается в два раза - геометрическая прогрессия.

\[p = nkT; \ p_o = n_okT \rightarrow n(h) = n_o e^{-\frac{m_0 g h}{kT}} = n_0e^{-\frac{Mgh}{RT}}\]

\[n(\vec{r}) = n_o e^{-\frac{U(\vec{r})}{kT}} \rightarrow m_o * n_o = \rho_o; n_o * n = \rho \Rightarrow \rho(h) = \rho_o e^{-\frac{Mgh}{RT}}\]

\[F_c \sim v^2\]

\[F_C = C_f \frac{\rho v^2}{2} S; \ C_f - \mbox{влияние формы}, \ S - \mbox{площадь поперечного сечения}\]

\[C_{f_{\mbox{сферы}}} \approx 0,47 \approx 0,5\]

\subsection{Уравнение Магестокса}

Полезно в аэродинамике.

Мы считаем, что газ и жидкость - сплошные среды. Мы не ухватываем отдельные молекулы.

Проблема: концентрация 

\subsection{Длина свободного пробега}
Проведем некоторую оценку длины свободного пробега:

Если мы говорим про любой газ, воспринимая его как набор каких-то частичек, каждая молекула так или иначе будет путешествовать по ломаным траекториям.

Хотелось бы узнать, какая длина пробега без ударения в другую молекулу - $\lambda$.

Будем предствалять молекулы как шарики. Для первой оценки будет этого достаточно. Тогда для того чтобы молекулы соударились, нужно, чтобы они соударились и расстояние между ними равно $d$.
 
Рассмотрим ситуацию, когда молекула летит с какой-то скоростью. Крайняя ситуация - молекулы "чиркуют". Площадь сечения равна $\pi d^2$.

При слуйчаном расположении молекул хотелось бы узнать 

Скорость хотелось бы взять относительной $\vec{g} = \vec{v_1} - \vec{v_2}$.

Пусть есть "облако" молекул.

Зная скорость шарика, мы можем найти скорость шарика.

$N = n S g dt \Rightarrow 1 = n S g \tau \Rightarrow \frac{1}{\tau} = \nu = n S g$

\[\nu = \pi d^2 g n\]
\[\nu = v \tau = \frac{v}{\nu} = \frac{v}{nSg}\]

\[\]