\section{Устройства ввода-вывода}

\subsection*{Микрофон / аналоговые колонки}

\subsection*{Видеокамера / монитор}

Нужно выяснить, как устроено зрение человека. Человек видит с помощью зрительного нерва. Глаз представляет собой некоторую линзу (хрусталик). Он приломляет изображение на зрительный нерв. Зрительный нерв представляет собой тысячи рецепторов и некоторых датчиков. Они снимает интенсивность света. Изображение проецируется на сетчатку в перевернутом виде(?). Мозг это изображение воспринимает и интерпретирует.

Когда придумали фотоаппарат (110 лет назад), сделали модель глаза. Есть объектив, который проецировал изображение. В фотоаппарате есть пластинка из солей серебра, они меняют оттенок в зависимости от освещенности

Изображение не прочное, его надо фиксировать - фотопластинка, потом пленка и фотобумага. Изображение закреплялось в бессветной среде.

У этого процесса есть недостатки: трата времени, бессветная среда.

Сделали пленку (светочувстительная кассета)

Нужно пленку проявлять - мокрые технологии. Эти технологии несколько десятилетий совершенствовались.

Из пленочного фотоаппарата была сделана кинокамера. Она часто снимала кадры (..?). - кинопроектор.


Все это было не очень удобно. Была придумана телевизионная камера. Некоторая электроника бегала по строкам матрицы и последовательно передавала по каналам связи код (сигналы) ТВ-станции, которая его отображала ПВР(?).

Телевизионная камера называлась иконоскопом. Его особенность - то, что он ничего не помнил. Датчики не были долговременными.

Цифровой фотоаппарат представлет собой некую версию иконоскопа, он имел объектив и он проецировал изображение на матрицу - цифровой светочувствительный миниатюрный датчик, твердотельный кристалл (полупроводниковый прибор?..). Есть градация света и интенсивности.

Датчик сам по себе не может (..?), цифоровой - может фиксироать изображение и последовательно передать миллионы пикселей (информацию об изображении) контроллеру (он отвечает он передает кадры какой-то другой системе). 

Достоинства: позволяет иметь изображение в цифровом виде (...) 

Цифорвая видеокамера: рекордер + камера = конкордер(?)

Веб-камера - цифровая видеокамера, рассчитанная на простую передачу изображения ()

Мониторы

Нам надо, чтобы компьютер что-то показал, чтобы пользователь что-то видел.

1. печать на бумагу (принтеры)

2. кинопроектор с киноэкраном: инерция зрения: 24 кадра в секунду - достаточное количество кадров для того чтобы считать изображение без рывков и т. п. (некоторые принципы, например, для анимации).

3. телевизор: про аналоговый - выпуск луча, вспихывает частица люминофора

4. монитор
\section{Устройства внешней памяти}

