\section{Основы трудового права}

%\begin{enumerate}
	%\item Предмет, принципы и источники трудового права
	%\item Трудовой договор (контракт): содержание и порядок заключения
	%\item Прекращение трудового договора. Юридические гарантии увольняемым работникам
%\end{enumerate}

\subsection{Предмет, принципы и источники трудового права}
Трудовое право - отрасль права, которая регулирует порядок возникновения, действия и прекращения трудовых отношений, определяет режим совместного труда работников, устанавливает меру труда, правила по охране труда и порядок рассмотрения трудовых споров.

Предметом трудового права являются трудовые отношения, возникающие при реализации работником своей способности к труду путем заключения трудового договора с работодателем, а также другие общественные отношения, связанные с трудовыми. К ним относятся:

\begin{enumerate}
	\item организационно-управленческие отношения профсоюзного органа, представляющего интересы работников, с администрацией предприятия по поводу улучшения условий труда, быта и отдыха:
	\item отношения, связанные с повышением квалификации работников;
	\item отношения по перераспределению рабочей силы;
	\item отношения по поводу занятости и трудоустройства;
	\item отношения, связанные с возмещением материального ущерба и др.
\end{enumerate}

Сторонами и субъектами трудового правоотношения являются работник и работодатель. Кроме сторон субъектами трудовых правоотношений признаются профсоюзы, иные представительные органы работников, представительные органы работодателей, в том числе руководители организаций.

\subsubsection{Основные принципы трудового права}
\begin{enumerate}
	\item Свобода труда, запрещение принудительного труда и дискриминации в сфере труда;
	\item Защита от безработицы и содействие в трудоустройстве, обеспечение права каждого работника на справедливые условия труда, равенство прав и возможностей работников;
	\item Обеспечение права каждого работника на своевременную и в полном размере выплату справедливой заработной платы, продвижение по работе с учетом производительности труда, квалификации и стажа работ, а также на профессиональную подготовку, переподготовку, повышение квалификации и др.;
\end{enumerate}

\subsubsection{Источники трудового права}
\begin{enumerate}
	\item Федеральные источники:
	\begin{itemize}
		\item Конституция РФ;
		\item Трудовой кодекс РФ;
		\item Федеральные законы, содержащие нормы трудового права;
		\item Указы Президента РФ, направленные на регулицию трудовых отношений;
		\item Постановления Правительства РФ, регулирующие трудовые отношения;
		\item Нормативные акты министерств и ведомств РФ;
		\item Постановления Пленума Верховного Суда РФ по спорным вопросам трудовых отношений (условно).
	\end{itemize}
	\item Локальные источники:
	\begin{enumerate}
		\item Нормативные акты субъектов Федерации;
		\item Правотворчество органов местного самоуправления;
		\item Правила внутреннего трудового распорядка, установленные в организации;
		\item Коллективные договоры и соглашения;
		\item Трудовые договоры (контракты);
		\item Приказы и распоряжения руководителей организации.
	\end{enumerate}
\end{enumerate}

Особое место среди источников трудового права занимают акты Международной организации труда (МОТ).

\subsection{Трудовой договор (контракт): содержание и порядок заключения}

Трудовой договор - это соглашение между работодателем и работником, в соответствии с которым работодатель обязуется предоставить работнику работу по предусловленной трудовой функции, обеспечить условия труда, своевременно и в полном размере выплачивать работнику заработную плату, работник обязуется \underline{лично} выполнять определенную трудовую функцию, соблюдать правила внутреннего трудового распорядка.

Сторонами трудового договора являются работодатель и работник, в нем указываются: фамилия, имя, отчество работника и наименование работодателя - юридического лица или фамилия, имя, отчество работодателя - физического лица.

В содержании трудового договора входят условия, определяющие права и обязанности работника и работодателя. Эти условия могут быть непосредственными (устанавливаемыми соглашением сторон) и производыми (обязательными в силу закона, и о них стороны не договариваются).

Условия трудового договора делятся также на существенные и дополнительные. 

Существенными условиями являются:
\begin{enumerate}
	\item место работы (с указанием структурного подразделения);
	\item дата начала работы;
	\item наименование должности, специальности, профессии с указанием квалификации в соответствии со штатным расписанием;
	\item права и обязанности работника;
	\item права и обязанности работодателя;
	\item характеристики условий труда, компенсации и льготы за работу в тяжелых, вредных или опасных условиях;
	\item режим труда и отдыха (если он в отношении данного работника отличается от общих правил, установленных в организации);
	\item условия оплаты труда;
	\item виды и условия социального страхования, непосредственно связанные с трудовой деятельностью;
\end{enumerate}

В трудовом договоре могут предусматриваться дополнительные условия: об испытании, о неразглашении охраняемой законом тайне (государственной, служебной, коммерческой и иной), об обязанности работника отработать не мнеее установленного срока, если его обучение производилось за счет работодателя, а также иные условия, не ухудшающие положение работника по сравнению с Трудовым кодексом и иными нормативными актами.

Условия трудового договора могут изменены только по соглашению сторон и в письменной форме. Это относится и к переводу на другую работу.

Перевод на другую работу - это постоянное или временное изменение трудовой функции работника или структурного подразделения, в котором он работает при продолжении работы у того же работодателя, а также перевод на работу в другую местность вместе с работодателем. Перевод работника без его согласия допускается только в исключительных случаях на срок до одного месяца.

Не требует согласия работника перемещения его у того же работодателя на другое рабочее место, в другое структурное подразделение, расположенное в той же местности, поручение ему работы на другом механизме или агрегате, если это не влечет за собой изменения условий трудового договора (ст. 72 ТК РФ).

Трудовые договоры могут заключаться:

\begin{enumerate}
	\item на неопределнный срок;
	\item на определенный срок (не более пяти лет; срочный трудовой договор). Он заключается в случае, когда трудовые отношения не могут быть установлены на неопределенный срок с учетом характера предстоящей работы.
\end{enumerate}

Если в трудовом договоре не оговорен срок его действия, то считается, что он заключен на неопределенный срок. В случае если ни одна из сторон не потребовала расторжения срочного трудового договора в связи с истечением его срока, а работник продолжает работу, трудовой договор считается заключенным на неопределенный срок.

Трудовой договор вступает в силу со дня подписания работником и работодателем, если иное не установлено нормативными актами или трудовым договором, либо со дня фактического допущения работника к работе. Заключение трудового договора допускается с лицами, достигшими 16 лет, а с согласия одного из родителей или попечителя и органа опеки и попечительства - с 14 лет.

Трудовой заключается в письменной форме, составляется в двух экземплярах, каждый из которых подписывается сторонами (один передается работнику, другой хранится у работодателя). Прием оформляется приказом (распоряжением) работодателя, который объявляется работником под расписку в трехдневный срок со дня подписания трудового договора. При приеме на работу работодатель обязан ознакомить работника с правилами внутреннего трудового распорядка и иными документами, имеющими отношение к трудовой функции работника. 

\subsection{Прекращение трудового договора. Юридические гарантии увольняемым работникам}

В соответствии со статьей 77 ТК РФ основанием прекращения трудового договора являются:

\begin{enumerate}
	\item соглашения сторон;
	\item истечение срока трудового договора за исключением случаев, когда трудовые отношения фактически продолжаются, и ни одна из сторон не потребовала их прекращения;
	\item расторжение по инициативе работника;
	\item расторжение по инициативе работодателя;
	\item перевод работника по его просьбе или с его согласия на работу к другому работодателю или переход на выборную должность;
	\item отказ работника от перевода на другую работу вследствие состояния здоровья в соответсвии с медицинским заключением или отсутствием у работодателя соответствующей работы;
	\item отказ работника от перевода в связи с перемещением работодателя в другую местность;
	\item обстоятельства, не зависящие от сторон;
	\item нарушения установленных правил заключения трудового договора, если оно исключает возможность продолжения работы и др.
\end{enumerate}

Работник имеет право расторгнуть трудовой договор, предупредив работодателя в письменной форме за две недели. По соглашению между работником и работодателем трудовой договор может быть расторгнут и досрочно.

Статья 81 ТК РФ перечисляет основания расторжения трудового договора по инициативе работодателя:
\begin{enumerate}
	\item ликвидация организации либо прекращение деятельности работодателем (юридическим лицом);
	\item сокращение численности или штата работников организации;
	\item несоответствие работника занимаемой должности или выполняемой работе вследствие недостаточной квалификации, подтвержденной результатом аттестации;
	\item неоднократное неисполнение работником без уважительных причин трудовых обязанностей, если он имеет дисциплинарное вызскание;
	\item однократное грубое нарушение работником трудовых обязанностей:
	\begin{itemize}
		\item прогул (отсутствие на рабочем месте более четырех часов подряд);
		\item появляение не работе в состоянии алкогольного, наркотического или иного токсического опьянения;
		\item разглашения охраняемой законом тайны, ставшей известной работнику в связи с исполнением им трудовых обязанностей;
		\item совершение по месту работу хищения (в том числе мелкого) чужого имущества, умышленного его уничтожения, растраты или повреждения;
		\item нарушение работником требований по охране труда, если это повлекло за собой тяжкие последствия либо создавало реальную угрозу таким последствиям;
		\item совершение виновных действий работником, обслуживающим денежные или товарные ценности, если это дает основание для утраты доверия к нему со стороны работодателя;
		\item совершение работником, выполняющим воспитательные функции, аморального проступка, не совместимого с продолжением данной работы;
		\item предстваление работником работодателю подложных документов или ложных сведений при заключении трудового договора(?) 
	\end{itemize} 
	\item
\end{enumerate}

\subsubsection{Юридические гарантии увольняемым работникам по инициативе работодателя}
\begin{enumerate}
	\item Допускается такое увольнение за исключением пункта 1 статьи 81 ТК в период заболевания работника или его отпуска;
	\item о предстоящем увольнении в связи с ликвидации организации, в сязи с сокращением численности или штат работники предупреждаются персонально не менее чем за два месяца до увольнения. Увольняемому выплачивается выходное пособие в размере среднего месячного заработка, а также за ним сохраняется средний месячный заработок на период трудоустройства, но не свыше двух (в исключительных случаях - трех) месяцев со дня увольнения с зачетом выходного пособия 
\end{enumerate}
