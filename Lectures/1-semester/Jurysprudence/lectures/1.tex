\section{Основы теории государства и права}

\subsection{Понятие государства. Понятие права, его нормы и источники. Система права}

Право происходит от слова "юстиция".

Право - совокупность общеобязательных правил поведения (норм права), устанавливаемых государством для регулирования наиболее важных общественных отношений.

Государство - территориально организованная общность населения, построенная на началах власти и действующая на основе права с помощью специального государственного аппарата.

В общей теории права понятие \textit{право} делится на два вида: субъективное и объективное.

Субъективное право - право конкретного лица, которое имеет возможность свободно выбирать для себя формы поведения в рамках закона. В субъективном смысле право выступает как мера свободы человека, мера дозволенного поведения личности.

Объективное право - система общеобязательных правил поведения, выраженная в законодательных актах, принимаемых или санкционированных государством. В объективном праве выделяют два вида: публичное и частное право.

Публичное право регулирует такие отношения, в которых одним из субъектов является государство. Публичному праву присущи императивные нормы, содержащие категорические, безусловно обязательные предписания поведению человека. Они делятся на обязывающие и запрещающие.

Частное право регулирует личные и групповые интересы. Для частного права характерны диспозитивные нормы, дающие простор инициативе и возможности заинтересованным лицам самостоятельно сделать выбор и определять свои отношения.

\subsubsection{Основные признаки права}
\begin{enumerate}
	\item Общеобязательность исполнения;
	\item Установление и гарантированность государством;
	\item Формальная определенность;
	\item Нормативность и системность (предстваляет собой систему норм);
	\item Многократность применения.
\end{enumerate}

\subsubsection{Основные функции права}
\begin{enumerate}
	\item Регулятивная: проявляется в регулировании общественных отношений;
	\item Охранительная: заключается в охране прав и законных интересов субъектов права;
	\item Воспитательная: воздействует на граждан в направлении выполнения законов.
\end{enumerate}

Норма права - основной структурный элемент права.

Нормы права - установленные государством общие правила поведения, регулирующие общественные отношения.
\subsubsection{Структура нормы права}
\begin{enumerate}
	\item гипотеза;
	\item диспозиция;
	\item санкция.
\end{enumerate}

Гипотеза указывает на круг лиц, которым адресована норма, а также на обстоятельства, при которых она реализуется.

Диспозиция - сами правило поведения, права и обязанности субъекта правовых отношений.

Санкция указывает на меры государственного принуждения, которые применяются к нарушителям правила поведения, установленного в диспозиции.

Нормы права получают свое выражение в источниках права. Их основные виды: нормативные акты, прецеденты, правовые обычаи, нормативные договоры.

Нормативные акты содержат общеобязательные правила поведения и издаются в основных органах государственной власти и управления. С их помощью устанавливаются новые или изменяются / дополняются уже действующие нормы права.

Нормативные акты делятся на законы и подзаконные акты.

Конституция является законом высшей юридической силы.

Судебные прецеденты - решения по конкретному делу, являющиеся обязательными для той же или низшей инстанции при решении аналогичных судебных дел.

Правовые обычаи - правила поведения, первоначально сложившиеся в процессе самой жизни, в результате их многократного и длительного применения, а затем санкционированные и взятые под защиту государством.

Нормативный договор - соглашение между равноправными субъектами права по поводу деятельности, представляющий их общий интерес, которое содержит в себе правовые нормы; является основным источником в международном праве. Внутри РФ наиболее широко используется для регулирования трудовых отношений. Нормативный договор складывается из норм, институтов и отраслей.

Правовой институт - совокупность взаимосвязанных норм, регулирующих качественно однородные общественные отношения.

Под отраслью права понимается совокупность взаимосвязанных правовых институтов, регулирующих относительно самостоятельную область общественных отношений.

Существуют отрасли права: конституционное, административное, гражданское, трудовое, семейное, уголовное и др.

Право существует в неразрывной связи и взаимодействии с другими социальными регуляторами, устанавливающие общие правила поведения. Это социальные нормы:
\begin{itemize}
	\item Обычаи и традиции;
	\item Нормы морали и нравственности;
	\item Нормы, принимающиеся партийными органами или общественными организациями;
	\item Религиозные.
\end{itemize}

\subsection{Понятие правоотношения. Правонарушение и юридическая ответственность}

Правовые отношения - общественные отношения между людьми, урегулируемые нормами права. Основными субъектами правовых отношений выступают физические лица (люди) и юридические лица (организации, обладающие определенными признаками).

Субъекты правоотношений имеют права и юридические обязанности. Они совершают деяния в форме действия или бездействия, которые могут быть правомерными и неправомерными.

Неправомерные деяния (правонарушения) - деяния, нарушающие установленный правопорядок.

Правонарушение - противоправное общественно опасное деяние, совершенное умышленно либо по неосторожности. За правонарушение законом прдусматриваются гражданская, административная, уголовная, дисциплинарная ответственность.

По степени общественной опасности правонарушения делятся на проступки и преступления.

Преступление признается общественно опасным деяние, запрещенное уголовным правом под угрозой наказания.

Проступки - правонарушения, не признанные преступлениями. Они характеризуются меньшей степенью общественной опасности.

Состав правонарушения - совокупность его обязательных элементов:
\begin{enumerate}
	\item объект правонарушения;
	\item объективная сторона правонарушения;
	\item субъект правонарушения;
	\item субъективная сторона правонарушения.
\end{enumerate}

Отступление одного из элементов означает отсутствие состава правонарушения.

Объект првонарушения - то, что охраняется право,, на что напрвлено правонарушение, чему причиняется вред.

Объективная строна правонарушения - само противоправное деяние, общественно опасные последствия, причинная связь между деянием и наступившими последствиями.