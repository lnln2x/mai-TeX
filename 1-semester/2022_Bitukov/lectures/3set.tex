\section{Мощность множества. Счетность множества рациональных чисел}

\subsection{Мощность множества}

\begin{definition}
	Множество $\mathbf{X}$ называется равномощным множеству $\mathbf{Y} (\mathbf{X} \thicksim \mathbf{Y})$, если существует взаимнооднозначное отображение $f: \mathbf{X} \rightarrow \mathbf{Y}$, так и однозначное отображение $f^{-1}: \mathbf{Y} \rightarrow \mathbf{X}$. Введенное отношение $\mathbf{X} \thicksim \mathbf{Y}$ является отношением эквивалентности.
\end{definition}

\begin{definition}
	Класс эквивалентности, которому принадлежит данное множество, называется мощностью этого множества. Если $\mathbf{X} \thicksim \mathbb{N}$, то оно называется счётным.
\end{definition}

\subsection{Счетность множества рациональных чисел}

\begin{definition}
	Множества, равномощные множеству натуральных чисел, называются счётными. $\mathbf{X} \thicksim \mathbb{N} \Rightarrow \exists f: \mathbb{N} \rightarrow \mathbf{X}$ - взаимнооднозначное $\Rightarrow \forall x \in \mathbf{X}$ $ \exists n: f(n) = x$. Обозначают $x_n = f(n)$.
\end{definition}

\begin{theorem}
	Множество $[0; 1]$ не является счетным.
\end{theorem}

\begin{proof}
	Множество ???
\end{proof}