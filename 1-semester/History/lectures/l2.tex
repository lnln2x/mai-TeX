\section{(18.11, 11:45)}

\subsection{Россия в 1914 - 1917 годах}
Россия промышленно слабо развитая страна, в годы войны почти вся промышленность стала работать на фронт, производство товаров широкого потребления резко сократилась $\rightarrow$ нарушение товарооборота между городом и деревней $\rightarrow$ деревня отказывается поставлять продовольствие в город, так как:
\begin{itemize}
	\item широкого потребления нет;
	\item деньги стремительно обесцениваются.
\end{itemize}

Кроме этого, "просели" \ железные дороги $\rightarrow$ низкая плотность $\rightarrow$ уже в 1914 году пошли сбои, а 1915 году с начала мобилизации промышленности начался кризис перевозок. Внимание: в целом продовольствие в стране имелось, так как в годы войны мы почти полностью прекратили экспорт зерна - примерно этот объем требовался городу и армии.

Ссудительный(?) фактор - буржуазия. Уже с 1915 года включается в мобилизацию промышленности, экономически усиливается и жаждет власти - в августе 1915 года в IV Госдуме формируется прогрессивный блок, который требует создание Правительства, ответственного не перед царем, а перед Думой. Ситуацию учиняет Гришка Распутин $\rightarrow$ 1915-1916 года $\rightarrow$ в министерстве чехарда: сменилось 4 главы правительства, 4 военных министра и 6 министров МВД, само правительство называли "кувырк-коллегией".

Итог: во второй пловине 1916 года буржуазия замысливает Дворцовый переворот с привлечением военных, начальник штаба ставки генерал Гусеев, командующий Юго-Западным фронтом генерал Брусилов, командующий Северным фронтом генерал Рузский. Кроме этого, переговоры ведутся с членами царской семьи о необходимости смещения Николая II на кого-нибудь из Романовых. Так в декабре 1916 года состоялось тайное совещание крупных промышленников и финансистов в Москве под руководством князя Львова - вступили в переговоры с дядей Николая II, великим князем Николаем Николаевичем.

Проблема: военные и Романовы пока колеблются $\rightarrow$ нужен катализатор.

Январь 1917 года: резкое ухудшение снабжения продовольствием Петрограда - даже введение продовольственных карточек не помогает - с одной стороны, на черном рынке дичайшая спекуляция, с другой - огромные очереди у продовольственных магазинов. Ситуация накалена.

18 февраля 1917 года - забастовка Путиловского завода (самый крупный завод Петербурга) $\rightarrow$ 20 февраля - завод полностью закрыт - 30 тысяч рабочих выброшены и по поводу прежник стачек понятно, что их поддержат и другие заводы.

Председатель октябрист Родзянко (крупная буржуазия) - с ним поддерживают контакт представители левых партий - меньшевики, эсеры и людовики. Они берут на себя организацию забастовки, они планируются провести 23 февраля (8 марта) с использованием голодных женщин.

Итог: 23 февраля - 128 тысяч госслужащих, 24 февраля - 218 тысяч госслужащих, 25 февраля бастуют уже 300 тысяч рабочих (80$\%$ от общего числа), 26 февраля производится переход восставших из петроградского гарнизона - если утром 27 февраля перешло 10 тысяч солдат, то 1 марта - 170 тысяч солдат из 180 тысяч имевшихся. Вишенка на торт - 28 февраля гражданские врываются в главный арсенал города и растаскивают более 70 тысяч огнестрельного оружия. 

Общий итог на 1 марта: по городу бегает более 300 тысяч рабочих, из 70 тысяч руженые + 170 тысяч солдат $\rightarrow$ ставка Главного командования, 27 февраля планировавшая послать на Петроград карательные войска, от своего замысла отказывается. Военные и буржуазия наконец-то состыкуются - Николая II, пытавшегося приехать в Царское Село, заворачивают на Псков - штаб Серевного флота, где он 2 марта отрекается от престола от себя и сына Алексея в пользу брата Михаила. Проблема для буржуазии - перестаралась: ситуация дестабилизировалась настолько, что буржуазия не в состоянии ее контролировать $\rightarrow$ власть в городе в руках рабочих и солдат, буржуазию не обожающих.