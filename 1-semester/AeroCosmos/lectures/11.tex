\section{Основные положения молекулярно-кинетической теории. Характерные скорости молекул (17.11)}
\subsection{Идеальный газ}
Пусть дан набор некоторых движимых молекул. Мы считаем, что эти молекулы непрерывно и хаотично двигаются, соответственно, между ними есть расстояние. Хотелось бы понять, как можно оценить то давление, которое будет оказываться на стенку (пусть) какой-то банки каким-то количеством газа (в котором есть какое-то количество молекул).

Рассмотрим газ, состоящий из одних и тех же молекул, соответственно, все молекулы имеют одинаковую массу, обозначим её $m_0$; эти молекулы будут, безусловно, так или иначе периодически ударяться о стенку. Можно сказать, что в результате этого соударения будет происходить изменение импульса. То есть если была какая-то скорость $v$, то был и импульс $p$, который направлен под произвольным углом (потому что молекула(?)).

Пусть молеуклы будут зеркально отражаться. Если молекула отлетела, то $p_y \rightarrow p_y$, то есть касательная компонента остается без изменений, а компонента $x$ поменяет знак на противоположный. Теперь мы можем вычислять изменение импульса (оно равно силе, действовашей во время акта соударения):

\[\frac{\triangle\vec{p}}{\triangle t} = \frac{d\vec{p}}{dt} = \vec{F} \ (\triangle p_x = -p_x - p_x = -2p_x)\]

На молекулу, которая, будучи во время взаимодействия со стенкой, была вблизи этой стенки, действует сила со стороны стенки. Но, по третьему закону Ньютона, сила, с которой молекула действует, равносильна силе, которая молекула получает в ответ. Тогда мы получаем ровно такую же силу по модулю, однако противоположную по направлению.

Тогда если мы говорим, что мы меняем компоненту $x$ скорости, то можно сказать, что есть компонента $x$ импульса:

\[\frac{\triangle p_x}{\triangle t} = F_x \Rightarrow F_x = \frac{-2px}{\triangle t}\].

Если интересует воздействие на стенку, то можно отбросить минус:

\[F = \frac{2px}{\triangle t}\]

Обозначим давление $P = \frac{F}{S}$. Если мы рассматираем площадку площадью $S$, стоит учитывать, что ударяется не только одна молекула в данный момент. Явно сталкивается какое-то большее количество молекул. Тогда обозначим общую силу $P_{\mbox{общ}} = N * \frac{2p_x}{\triangle t}$. Хотелось бы оценить, можно ли взять в расчет эту величину. Введем понятие концентрации - количество рассматриваемых молекул на единицу объема: возникает проблема в том, чтобы понять, какой объем взять. 

Пусть есть некоторый слой молекул около площадки, мы понимаем, что не все из них смогут взаимодействовать с ней, так как некоторые другие молекулы отскочили от стенки. В лучшем случае соприкоснутся со стенкой $\frac{1}{2}$ всех молекул слоя. Тогда общая сила за время $\triangle t$ равна

\[F_{\mbox{общ}} = \frac{1}{2} * n * S * V_x * \triangle t * \frac{2p_x}{\triangle t}\]
